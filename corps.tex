% ======================================================================
% ===================== CONTENT OF COVER LETTER ========================
% ======================================================================

Madame, Monsieur,

Depuis toujours, l’univers de la parfumerie me fascine. Derrière chaque création se cache une
précision presque scientifique, une vision artistique singulière, et chez Parfums Christian
Dior, une empreinte française unique, à la fois audacieuse et intemporelle. Plus qu’un savoir-
faire, c’est une signature de style et de constance, une forme d’excellence incarnée, qui me
donne aujourd’hui envie de mettre mes compétences logistiques au service de votre Maison.

Actuellement en stage chez Nestlé au sein de la cellule don, j’assure la coordination
quotidienne de la redistribution des dons alimentaires entre les entrepôts et les associations
partenaires. J’y gère l’extraction et l’analyse de données (SAP, Excel), l’alimentation de
reportings Power BI, ainsi que le suivi rigoureux des flux et du planning logistique. Ce poste
m’a permis de structurer ma méthode de travail, de renforcer ma rigueur et de développer un
sens pointu de l’anticipation, dans un environnement exigeant et en constante évolution.

En parallèle de mes études, j’ai toujours travaillé, notamment dans le retail, ce qui m’a appris
à évoluer avec agilité entre différents environnements, à gérer les imprévus et à garder une
approche orientée client, quelle que soit la mission. Ces expériences, tout comme mon
échange universitaire dans un contexte multiculturel, ont renforcé ma capacité à collaborer
avec des profils variés, à communiquer avec clarté et à m’adapter rapidement aux dynamiques
d’équipe.

Ce stage chez Parfums Christian Dior représente pour moi bien plus qu’une ligne sur un CV :
c’est la possibilité de contribuer concrètement à la transformation logistique d’une Maison
dont j’admire profondément la culture, la modernité et la force créative. J’y vois une chance
de relier mes compétences analytiques et opérationnelles à un univers que je respecte et que je
comprends profondément.

Je me tiens à votre disposition pour échanger de vive voix sur ma motivation et sur la façon
dont je pourrais accompagner vos projets.

Veuillez agréer, Madame, Monsieur, l’expression de mes salutations distinguées.

\flushright{Eucira KASSANJI}